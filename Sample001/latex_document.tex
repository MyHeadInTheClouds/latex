% A percent sign tells the LaTeX compiler to ignore whatever comes after it.
% So you can use it to make comments in your .tex file.


% We have to tell LaTeX what sort of document we're creating.
% This will be an article, and we're using 11pt font (10 is too small, 12 is too big, imho).
\documentclass[11pt]{amsart}

% This next line forces the compiler to make pdf files. 
\pdfoutput=1

% Next we can include some packages with other features. 
% I have added one that makes your document take up a full page.
% The packages amsmath and amssymb give you a bunch of extra math symbols.
\usepackage{fullpage}
\usepackage{amsmath, amssymb}


% The following code defines a ``problem'' environment that I use for homeworks and exams.
% Check out the examples below.
\newtheorem{problem}{Problem}


% You can also define your own macros to save on typing time.
% The following macro allows me to type $\ZZ$ instead of 
% $\mathbb{Z}$ in order to produce the symbol for the integers.
\newcommand{\ZZ}{\mathbb{Z}}



%%%%%%%%%%%%%%
%%%%%%%%%%%%%%

% These lines put a heading on the first page. 
\title{Math 321 LaTeX Sample Document}
\author{Jenny K. Student}

% If you don't include a date between the braces, it will simply print today's date.
\date{}

%%%%%%%%%%%%%%

% Now we have to tell the compiler that the content of the article is beginning. 
% We also have to tell it to make a title and add our author info.
\begin{document}
\maketitle

Here is some sample text to show you what LaTeX does. 

To start a new paragraph, you need to leave a line of white space in your tex file.

To include math, you have two options. The first is called ``in line,'' and you do this by putting your math between two dollar signs (\$). For example, Fermat's Little Theorem tells us that if $p$ is a prime and $a$ is an integer such that $p \nmid a$, then $a^{p-1} \equiv 1 \pmod{p}$. Note that the exponent on $a$ has to go in braces (in the tex file). 

But if you really want an equation (or congruence) to stand out, then you ``display'' the math. It looks like this:
\[
	a^{p-1} \equiv 1 \pmod{p}.
\]
Nice, right?
Finally, we can make things look really nice for homeworks as follows:

\begin{problem}
	Let $X$ and $Y$ be blah blah blah \ldots
\end{problem}

\begin{problem}
	Let $X$ and $Y$ be as in the previous problem. What is $Z$?
\end{problem}

Notice that LaTeX automatically numbers the problems for us. 

Here is an Integral
$$\int_{a}^{b} x^2 dx$$

Sum $\sum_{n=1}^{\infty} 2^{-n} = 1$ inside text

Product $\prod_{i=a}^{b} f(i)$ inside text
$$\prod_{i=a}^{b} f(i)$$


% To tell the compiler when to stop working, you have to include the following line at the end.
\end{document}
